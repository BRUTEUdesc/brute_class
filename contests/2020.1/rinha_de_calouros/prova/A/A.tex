\section{Problema A: Mesada Par}
\vspace{-0.52cm}
\noindent \begin{verbatim}Arquivo: A.[c|cpp|java|py]                           Tempo limite: 1 s
\end{verbatim}

Omar é um garoto muito esperto e generoso da cidade EntreVila. Como ele sempre tem boas notas e é obediente, recebe mesada de seus pais. Por ser generoso, ele não fica com toda a mesada pra ele. Omar, as vezes, pega uma parte e doa para uma instituição de caridade da cidade. Mas, por outro lado, ele não é tão generoso assim, pois a parte que ele doa é a parte que o incomoda na mesada, ele só gosta de números pares. Se a mesada não é um valor par, ele doa a menor parte que ele pode tal que a mesada fique par, se a mesada é par ele não doa nada e guarda tudo.\\
Omar mora com os pais e nunca sai de casa, portanto não tem gastos, guarda toda a mesada(fora a parte doada). Ele anotou as mesadas que seus pais o deram no ano passado e quer saber quanto guardou, no total, neste tempo. Como os números podem variar bastante, ele pediu sua ajuda. Ele pediu para você escrever um programa que, de acordo com as mesadas do ano, diga quanto guardou. Lembre que ele sempre coloca valores pares em seu cofre.\\

A mesada de Omar não é pré-determinada nem tem um padrão necessariamente, segundo seus pais ela pode variar de 1 até 10^{9} reais.

\subsection*{Entrada}

A entrada consiste em 12 linhas cada uma com um inteiro $N$ representando a mesada daquele mês.

\subsection*{Restrições}
\begin{itemize}
    \item $1 \le N \le 10^{9}$.
\end{itemize}

\subsection*{Saída}

Você deve dar como saída o valor total acumulado por Omar no ano passado.

\begin{flushleft}
\begin{tabularx}{1.01\textwidth}{ | p{6cm} | p{10cm} | }
\hline
\textbf{Exemplo de Entrada} & \textbf{Exemplo de Saída} \\
\verbatiminput{./M/sample_1.in}
&
\verbatiminput{./M/sample_1.out}
\\
\hline
\verbatiminput{./M/sample_2.in}
&
\verbatiminput{./M/sample_2.out}
\\
\hline
\end{tabularx}
\end{flushleft}
