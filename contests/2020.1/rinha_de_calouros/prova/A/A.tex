\section{Problema A: Mesada Par}
\vspace{-0.52cm}
\noindent \begin{verbatim}Arquivo: A.[c|cpp|java|py]                           Tempo limite: 1 s
\end{verbatim}

Omar é um garoto muito esperto e generoso da cidade EntreVila. Como ele sempre tem boas notas e é obediente, recebe mesada de seus pais. Por ser generoso, ele não fica com toda a mesada pra ele e doa para uma instituição de caridade da cidade. Porém ele não é tão generoso assim já que a ele doa varia de acordo com uma condição: ele só gosta de números pares. \\
Se a mesada não é um valor par, ele doa a menor parte que ele pode tal que a mesada fique par, já se a mesada é par ele não doa nada e guarda tudo em sua poupança. \\
Omar mora com os pais e nunca sai de casa, portanto não tem gastos e guarda tudo que sobra da mesada. \\
Em seu caderninho ele anotou as mesadas que seus pais o deram no ano passado mas esqueceu de anotar quanto ele realmente guardou. Como as habilidades matemáticas dele se limitam a saber se um número é par ou não, ele pediu para você escrever um programa que de acordo com as mesadas do ano diga quanto conseguiu acumular. \\

A mesada de Omar não é pré-determinada nem tem um padrão necessariamente, segundo seus pais ela pode variar de 1 até 10^{9} reais.

\subsection*{Entrada}

A entrada consiste de um caso único de teste. O caso de teste contém 12 linhas, cada uma com um inteiro $N$ representando o valor da mesada recebido de seus pais naquele mês.

\subsection*{Restrições}
\begin{itemize}
    \item $1 \le N \le 10^{9}$.
\end{itemize}

\subsection*{Saída}

Você deve dar como saída o valor total acumulado por Omar no ano passado.

\begin{flushleft}
\begin{tabularx}{1.01\textwidth}{ | p{6cm} | p{10cm} | }
\hline
\textbf{Exemplo de Entrada} & \textbf{Exemplo de Saída} \\
\verbatiminput{./M/sample_1.in}
&
\verbatiminput{./M/sample_1.out}
\\
\hline
\verbatiminput{./M/sample_2.in}
&
\verbatiminput{./M/sample_2.out}
\\
\hline
\end{tabularx}
\end{flushleft}
