\section{Problema M: MMM}
\vspace{-0.52cm}
\noindent \begin{verbatim}Arquivo: M.[c|cpp|java|py]                           Tempo limite: 1 s
\end{verbatim}

Juninho é um rapaz, digamos assim, gulosinho. Não somente por adorar resolver problemas que envolvem soluções $greedy$, mas também por adorar MMM's (Migalhas Mega Maravilhosas). \\
Apesar de guloso, Juninho é extremamente preocupado com sua dieta. Para isso precisa de sua ajuda para desenvolver um algoritmo o ajude a acompanhar como anda seu consumo de MMM's. \\

O algoritmo tem como objetivo informar quantos MMM's Juninho anda comendo em média por dia e qual é o sabor de MMM que Juninho mais gosta.

\subsection*{Entrada}

A primeira linha consiste de dois inteiros $S$ e $D$ que representam, respectivamente, a quantidade de sabores diferentes de MMM's e a quantidade de dias. \\
As próximas $D$ linhas contém $S$ inteiros. Sendo $X_i$ a quantidade de MMM's do sabor $i$ consumida naquele dia.

\subsection*{Restrições}
\begin{itemize}
    \item $1 \le S,D \le 10^{3}$.
    \item $0 \le X_i \le 100$.
    \item $1 \le i \le S$.
\end{itemize}

\subsection*{Saída}

Você deverá imprimir duas linhas. A primeira linha informa quantos MMM's Juninho anda comendo em média por dia (Juninho só come MMM's inteiras) e a segunda linha informa qual seu sabor preferido (é garantido que ele sempre terá um preferido).

\begin{flushleft}
\begin{tabularx}{1.01\textwidth}{ | p{6cm} | p{10cm} | }
\hline
\textbf{Exemplo de Entrada} & \textbf{Exemplo de Saída} \\
\verbatiminput{./M/sample_1.in}
&
\verbatiminput{./M/sample_1.out}
\\
\hline
\verbatiminput{./M/sample_2.in}
&
\verbatiminput{./M/sample_2.out}
\\
\hline
\end{tabularx}
\end{flushleft}
